\chapter{Kurzanalyse}
\label{chapter-analyse}

Für die Entwicklung eines Dating-Protokolls ist es zwingend notwendig eine Analyse der geistig verwirrten Hobby-Coaches durchzuführen.
Dabei soll verstanden werden, was der arme Timmy in den nächsten Stunden seines Dating-Coaching durchmachen muss und welche groben Fehler seitens der Coaches begangen werden.
Wer an dieser Stelle einen Methodik-Teil erwartet hat, wird ihn nicht bekommen.
Schließlich muss sich auf das subjektive Niveau der YouTuber begeben werden, um diese analysieren zu können.
Im zweiten Abschnitt wird eine Marktanalyse mit random Facts getätigt.

\section{Die Youtube-Szene}
\label{chapter-analyse-yt}

Die Plattform Youtube bietet Menschen die Möglichkeit, Videos anzusehen, zu bewerten, zu kommentieren und hochzuladen. 
Diese hat eine sehr große Reichweite, sodass folglich auch für das Dating Videos produziert, hochgeladen und konsumiert werden.
Es kann jedermann zu jeder Zeit ein Video zu einem beliebigen Thema veröffentlichen.
Hierbei geht es oftmals nicht um die Verbreitung von Inhalten und Wissen, sondern um die Anzahl der Klicks, die durch das Video generiert werden.
Schließlich stehen die Einnahmen im Vordergrund.
Damit einher geht, dass der Suchalgorithmus primär die Videos vorschlägt, die viele Aufrufe haben.
Verwendet man nun den Suchbegriff \glqq Frauen überzeugen\grqq~fällt eine besondere Technik auf – das Clickbaiting.
Die Abbildung \ref{fig:analyse-clickbait} zeigt den ersten Suchvorschlag des Algorithmus, wobei das Clickbaiting deutlich wird.
In dem Fall werden vorrangig gutaussehende Frauen mit großen Überschriften und teils mit weitem Ausschnitt oder in einer sexualisierten Pose dargestellt. 

\begin{figure}[h]
    \centering
    \includegraphics[scale=0.4]{Sources/clickbait.png}
    \caption{Klassisches Thumbnail mit Clickbaiting-Komponenten \cite{royalflushseduction}}
    \label{fig:analyse-clickbait}
\end{figure}

Als Nächstes werden die konkreten Inhalte analysiert.
Das beispielhafte Video von \glqq Flirt-Forschung\grqq, natürlich ein männlicher YouTuber, erläutert die Einteilung der Frauen in drei Gruppen \cite{royalflushseduction}.
Diese hängt davon ab, wie viele Alternativen die Frau neben dem Mann hat.
Je weniger Alternativen in Form von Männern eine Frau hat, desto einfacher soll das Dating sein.
Im Falle von vielen Alternativen muss der Typ überzeugend sein, um eine Chance zu haben.
Das heißt für den armen Timmy, dass er keine Probleme beim Dating hat.
Der arme Timmy hat nämlich einen großen Penis und ein Sixpack.

Würde der arme Timmy jedoch tiefer in die Materie eindringen, stößt er auf besonders fragwürdige Coaches.
Der Youtube Kanal \glqq sprich sie an!\grqq~erklärt den Zuschauern, wie sie innerhalb von 3 Minuten eine Frau tagsüber auf der Straße küssen können \cite{youtube_2019}.
Dabei spricht der Typ eine Frau an, welche aus Kolumbien ein paar Tage in Berlin ist.
Ein Kameramann filmt währenddessen von hinten die Flirtaktion beziehungsweise für den Zuschauer das Coaching.
Nachdem der YouTuber die Frau gegrüßt hat, berührt er die Frau zunehmend am Oberkörper.
Anschließend sagt er, dass ihm ihr Schnitzel gefällt, was er metaphorisch als Sexualisierung für das weibliche Hinterteil ausdrückt.
Später küsst er sie bei der Abschiedsumarmung auf den Mund und betont erneut, dass er in ihr Schnitzel beißen möchte. 
An der Stelle lehnt sie erst ab.
Trotzdem ignoriert er ihre Kommentare bewusst und redet weiter auf sie ein, sodass die Nummern ausgetauscht werden.
Generell empfiehlt der YouTuber, dass man möglichst weit gehen und das Maximum herausholen soll.
Das sollte kein Vorbild für den armen Timmy sein.

Ein weiterer YouTuber erläutert, dass man seine Finger von Frauen lassen solle, die sich in Therapie begeben \cite{markopua_2019}.
Als abschließendes Beispiel präsentiert der Ehrenmann Pimp Daniel eine Anleitung, wie man Nudes von einer Frau bekommt \cite{cobertv_2020}. 
Essenziell dafür ist ein \glqq Cockbild\grqq~trotz Verneinung zu senden.
Es ist also Bettzeit für den armen Timmy, das reicht mit dem Internet!   


\section{Marktanalyse Singlebörsen}

Inzwischen gibt es in Deutschland über 2500 Singlebörsen auf denen Singles, darunter unser kleiner Timmy, ihre Partnersuche bestreiten \cite{kundler}.
Zu diesen gehören Plattformen wie Tinder, Bumble, Fremdgehen69 und Paarship.
Der geschätzte Umsatz im Bereich der Singlebörsen wird für dieses Jahr (2022) auf knapp 97 Millionen Euro geschätzt \cite{statista}.
Der Umsatz soll laut Prognose weiter steigen und im Jahr 2027 auf 102,5 Millionen Euro steigen.
Anhand der in Abbildung \ref{fig:analyse-alter} dargestellten Altersstruktur der Nutzenden lässt sich ableiten, dass dem armen Timmy die Singlebörsen verwehrt bleiben.

\begin{figure}[h]
    \centering
    \includegraphics[scale=0.65]{Sources/alter.png}
    \caption{Nutzende nach Alter in Deutschland aus Oktober 2021 \cite{statista}.}
    \label{fig:analyse-alter}
\end{figure}

In dieser Hinsicht scheinen junge Erwachsene (18-24 Jahre) recht gut abzuschneiden.
Das ist weniger verwunderlich, da dieses Alter viele soziale Ausflüge, beispielsweise Dorfpartys und Shishabars, zulässt.
Den Hauptteil bildet die Altersgruppe von 25 bis 34 Jahre.
Dies lässt sich ebenso leicht erklären, schließlich hat der Zahn der Zeit an der Jugendliebe gekratzt.
Man erkennt den Partner kaum noch mehr wieder und schaut sich nach Alternativen um.
Ein leichter Rückgang ist im Alter von 35 bis 44 zu erkennen.
Dabei handelt es sich um Menschen, die sich mit 20 Jahren verheiratet haben.
Diese merken nun auch, dass der Partner nicht mehr so frisch ist, mussten aber aufgrund der Ehe\footnote{Steuerliche Vorteile} und dem gesellschaftlichen Druck noch eine Weile durchhalten.
Ist dieser Punkt allerdings, warum auch immer, überstanden, fängt das gefährliche Alter von 45 bis 54 Jahre an.
Schätzungen zufolge soll diese Altersgruppe sehr anfällig für Seitensprünge sein \cite{seitensprung}.
Den kleinsten Anteil machen die 55- bis 64-Jährigen aus.
Hier verirrt sich gelegentlich eine Person aufgrund der Midlife-Crisis auf eine Dating-Plattform.
Ab dem Alter von 65 Jahren gibt es offensichtlich keine Nutzer mehr.
Dies beruht auf mindestens einem der folgenden Gründe:
\begin{itemize}
    \item Das Altersheim bietet genug Auswahl
    \item Beide bereits gestorben
    \item Defizite in der Handhabung mit Technik
    \item Der demente Partner vergisst sowieso innerhalb von einer halben Stunde, wenn Uschi da war.
\end{itemize} 
Nach Statista betrugt im Oktober 2021 der Anteil an Nutzerinnen der Singlebörsen in Deutschland nur 30,5 \% \cite{statista}.
Verwunderlich ist dies wohl kaum, da nur 22 \% Pferdemänner\footnote{Nicht zu verwechseln mit Deckhengsten vom Fickstutenmarkt} den Reitsport ausmachen \cite{bliemel}.

Für den armen Timmy ist demnach nichts dabei.
Das ist auch gut so, schließlich wird der kleine Racker im November erst 14 Jahre alt.
Die Altersgruppen scheinen alle ihre Probleme zu haben, wobei die 50-Jährigen am besten abschneiden.
Generell verwenden alle volljährige Altersgruppen Singlebörsen\footnote{bis zu einem gewissen Alter}. 
Daher ist die wichtigste Anforderung an das zu entwickelnde Protokoll, dass es zeitlos und allgemeingültig ist.
Zudem ist in Abschnitt \ref{chapter-analyse-yt} deutlich geworden, dass Dating-Coaches auf YouTube sexistischer als Mario Barth sind – das geht besser.
