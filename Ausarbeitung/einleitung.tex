\chapter{Einleitung}
\label{chapter-einleitung}

In der heutigen Zeit suggerieren uns Online-Coaches auf Internetplattformen, beispielsweise YouTube, die Leichtigkeit des Datings.
Zahlreiche Anleitungen wie \glqq So wirst du zum Traummann\grqq, \glqq Tricks, mit denen du jede Frau ins Bett bekommst\grqq~und \glqq Psychologische Tricks, die dich unwiderstehlich machen\grqq~kursieren auf diesen Plattformen.
Diese recht subjektiven Videos wird nun der schüchterne Timmy (13 Jahre alt) anschauen, in der Hoffnung, seinen Crush in der Parallelklasse anzuschreiben.
Dabei wird der YouTube Algorithmus ihm Videos anzeigen, die nicht nur Clickbait-Potenzial haben, sondern womöglich auf seine Altersgruppe und sein Umfeld nicht zugeschnitten sind.
Seltsamerweise sind die Coaches meist weiße Männer aus der Mittelschicht, die dem armen Timmy vorgeben, wie er sich zu verhalten hat und welche Tricks er dabei anwenden muss.
%Dabei ist es als problematisch anzusehen, dass jede Person Inhalte hochladen kann und die vermeintlichen Richtlinien vom armen Timmy übernommen werden.
%Diese Inhalte werden zum Thema Dating werden inhaltlich nicht sinnvoll geprüft, weshalb primär nur die Anzahl an Klicks über die Verbreitung des Videos entscheidet.




\section{Beiträge dieser Arbeit}

In dieser Arbeit wird ein universelles Dating-Protokoll vorgestellt, mit der du jede Frau für dich gewinnst.
Dieses ist zeitlos und kann von jeder Altersgruppe angewendet werden.
Im Gegensatz zu jedem Amateur-YouTube-Coach habe ich die wahre Essenz des Datings erforscht und habe die Psychologie der Frauen vollkommen durchdrungen.
Das Protokoll liefert insbesondere für das Online-Datingprofil genaue Richtlinien und beschäftigt sich mit diversen Anwendungsgebieten.
Professionelle Hilfe durch Herrn Professor Doktor Love und Beziehungsexperte Professor Doktor Tom Klatt geben einen Einblick in die brisanten Themen des Erscheinungsbildes und die erfolgreiche Beziehung.



\section{Aufbau dieser Arbeit}

Die Arbeit beginnt mit einer Kurzanalyse (siehe Kapitel \ref{chapter-analyse}), welche die YouTube-Szene und den Singlebörsen-Markt beleuchtet.
In Kapitel \ref{chapter-protokoll} wird ein konkretes Dating-Protokoll vorgestellt.
Abschließend wird eine Zusammenfassung der Arbeit und ein Ausblick für nachfolgende Arbeiten in Kapitel \ref{summary} präsentiert.