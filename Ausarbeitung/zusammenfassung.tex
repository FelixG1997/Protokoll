\chapter{Zusammenfassung und Ausblick}
\label{summary}


Nachdem die Menschheit jahrelang im Dunkeln des Datings getappt ist, wurde nun der Meilenstein in Form dieses Unikats vollendet.
Das zeitlose und generalisierte Dating-Protokoll liefert generationsunabhängige Anleitungen und Richtlinien für komplexe Bereiche des Datings.
Für das Online-Dating wurde der Fokus auf das Profil gelegt und im Anschluss ein Chatprotokoll entwickelt.
Das nachfolgende Kapitel zum Thema Erscheinungsbild, welches in enger Zusammenarbeit mit Herrn Professor Doktor Love erarbeitet wurde, befasste sich mit der Person selbst, dem passenden Umfeld und der Kleiderwahl.
Im Anschluss wurde das in der Kritik stehende Arbeitsumfeld im Bereich des Datings behandelt.
Dabei wurde eine konkrete Lösungsstrategie mitsamt einer Risikominimierung ausgefeilt, um nun doch die Kollegin zu daten.
Zuletzt wurden in Kooperation mit Professor Doktor Tom Klatt wertvolle Richtlinien für den Erfolg der Beziehung, die offensichtlich unmittelbar auf die Anwendung des Dating-Protokolls folgt, entwickelt.
Timmy hat durch das Dating-Protokoll viel gelernt und hat im Zuge der Anwendung des Protokolls eine neue Freundin (Bente, 12 Jahre alt).
\glqq Vielen Dank, du geschätzter und großartiger Frauenflüsterer!\grqq
   
Das Dating-Protokoll liefert bisher nur einen Guide für Männer, um Frauen zu gewinnen.
Im Zuge meines aktuellen Forschungsstandes ist eine Version für Frauen durchaus denkbar.
Die Nachfrage sollte immens sein.
Das Dating-Protokoll hat bewusst Fetische für die Allgemeingültigkeit ausgelassen. 
Es wäre jedoch spannend zu sehen, welche Techniken sich für dieses Interessengebiet anwenden lassen.
Für die Vollständigkeit stehen bestimmte Themen für zukünftige Arbeiten aus.
Dazu zählen: One-Night-Stands, Partys, das erste Date, Die Ex zurückgewinnen, die Frage nach einer Beziehung und die Liebe gestehen.
Sicherlich ist auch spannend, dass Finn-Christian dieses Werk als Premiere im Zuge seines Geburtstages lesen darf.
Somit bleibt abzuwarten, ob dieser trotz seines Beziehungsstatus einen Nutzen im Protokoll finden kann.
Offensichtlich ist hierfür der Abschnitt \ref{chapter-main-weisheit} ideal.

